\documentclass[a4paper,12pt]{article}

\usepackage[T2A]{fontenc}
\usepackage[utf8]{inputenc}
\usepackage[english,russian]{babel}
%\languageattribute{russian}{ancient}
\usepackage{indentfirst}
\usepackage{amsmath,amsfonts,amsthm,amssymb,mathtools}



\usepackage[left=2cm, right=1cm]{geometry}
\usepackage{fancyhdr}
\pagestyle{fancy}


\fancyhead[r]{\it Received 07.02.2010}
\fancyhead[l]{\it Ernesto Verda}
\fancyhead[c]{Scripta Uporologia 1, 1--2 (2014) }


%opening

\title{Популярное изложение положений квантово-газовой теории упоротости}
\author{Ernesto Verda}
\date{Принято в редакцию 07.02.2010\\ Опубликовано 02.06.2014}

\begin{document}

\maketitle

\begin{abstract}
Квантово-газовая теория упоротости является сравнительно новым направлением исследований, поэтому до сих пор сведения о ней фрагментарны. Задачей данной работы является популярное изложение основных положений данной теории, понятное даже неспециалисту. Будут рассмотрены механизмы возникновения упоротости, а также способы терапии адекватности.
\end{abstract}

\section{Основные механизмы возникновения упоротости}

Состояние упоротости возникает при образовании в мозге человека упорина. Упорин — вещество с химической формулой $CHRn$. При нормальных условиях упорин — бесцветный газ, не имеющий вкуса и запаха. При температуре $-162^\circ$C упорин превращается в жидкость, а при $-586^\circ$C затвердевает.

При температуре окружающей среды выше $30^\circ$C молекула упорина распадается с образованием одного упорона. Упорон — частица со спином $\frac{1}{3}$, вызывает образование 0,72 молекул упорина. При этом между радоном и углеродом возникает упоронная связь. Остальная энергия выделяется в виде тепла. Забегая вперёд, скажем, что излишнее образование тепла является основной проблемой при терапии адекватности.

При распаде упорина в мозге человека часть упоронов излучается через глазницы, т. к. костная ткань черепа является непреодолимым препятствием для упоронов. Оставшиеся упороны идут на образование новых молекул упорина. Отношение количества упоронов, образующих упорин не покидая мозга, к количеству излучённых упоронов называется коэффициентом упарываемости. Коэффициент упарываемости анатомически обусловлен и индивидуален для каждого человека.

Распад упорина происходит по линейному закону, т. е. количество испускаемых упоронов постоянно и не зависит от концентрации упорина. За одну секунду распадается 2856 молекул упорина. Количество образуемых при распаде упорина упоронов $u=2856\cdot 0,72=2056,32$ называется упориновой константой. Построим математическую модель процесса возникновения упоротости. Пусть $p_{u}(t)$ -- функция, показывающая зависимость количества упорина от времени. Тогда, исходя из изложенного выше
\begin{equation}
\dot{p}_{u}(t)=0,72s_{u}(t)-uk,
\end{equation}
где $k$ -- коэффициент упарываемости, а $s_{u}$ -- упороновый поток, количество упоронов, поступающих извне.

Напомним, что состояние упоротости возникает при образовании молекул упорина, и не зависит от его концентрации. Уровень упоротости измеряется во вланах и расчитывается по формуле

\begin{equation}
L_{u}=e^{0,72s_{u}-uk}
\end{equation}

Таким образом, упоротостью равной 1 влану называется такое сочетание концентрации упорина и упоронового потока, при котором концентрация упорина отаётся постоянной. Нетрудно рассчитать, что при $s_{u}=\frac{uk}{0,72}$ упоротость $L_{u}=1$, а при отсутствии внешнего упоронного потока $L_{u}=e^{-uk}$.

\section{Терапия адекватности}

Адекватностью называется болезненное состояние психики, когда пациент начинает оперировать в соответствии с законами аристотелевой логики. С развитием квантово-газовой теории упоротости, медицина получила мощный инструмент для терапии данного заболевания.

Терапия производится врачом-упорологом. В первую очередь необходимо определить коэффициент упарываемости пациента. Для этого производится внутривенная инъекция хлорида упорина и измеряется упороновый поток из глаз пациента. По количеству излучаемых упоринов можно судить о коэффициенте упарываемости. Затем пациент помещается под излучение стационарного генератора упоронов. В рабочей камере генератора производится сжигание триупорилбензола, что обеспечивает стабильный и хорошо управляемый поток упоронов. На данном этапе производится создание в мозге пациента запаса упорина. Для этого в течение некоторого времени поддерживается уровень упоротости больше единицы и производится контроль температуры мозга пациента. Процедура прекращается, когда количество упорина станет достаточным для поддержания анатомически обусловленного уровня упоротости в течение 6-12 месяцев. Если же таким образом не удаётся достичь достаточного уровня упоротости, то рассматривается вопрос о вживлении контейнера с хлоридом упорина прямо в череп пациента.

\end{document}
